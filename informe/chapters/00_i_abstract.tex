%Two resources useful for abstract writing.
% Guidance of how to write an abstract/summary provided by Nature: https://cbs.umn.edu/sites/cbs.umn.edu/files/public/downloads/Annotated_Nature_abstract.pdf %https://writingcenter.gmu.edu/guides/writing-an-abstract
\chapter*{\center \Large  Abstract}
%%%%%%%%%%%%%%%%%%%%%%%%%%%%%%%%%%%%%%
% Replace all text with your text
%%%%%%%%%%%%%%%%%%%%%%%%%%%%%%%%%%%

El presente informe tiene como objetivo resumir los principales aportes de la investigación desarrollada por David Harvey y Joris van der Hoeven, presentada el 28 de noviembre de 2020 en HAL Open Science bajo el título Integer Multiplication in Time O(n log n). Dicho estudio se enfoca en los antecedentes y la formulación de un nuevo algoritmo diseñado para reducir la complejidad computacional del problema de la multiplicación de enteros grandes, un desafío que ha sido objeto de estudio desde hace décadas. Durante mucho tiempo, se creyó que no habría avances significativos en este campo, lo que llevó a un estancamiento en la investigación. Sin embargo, el trabajo de Harvey y van der Hoeven representa un importante punto de inflexión en esta área.

%%%%%%%%%%%%%%%%%%%%%%%%%%%%%%%%%%%%%%%%%%%%%%%%%%%%%%%%%%%%%%%%%%%%%%%%%s
~\\[1cm]
\noindent % Provide your key words
\textbf{Keywords:} integer multiplication, algoritmo, complejidad computacional, investigación

\vfill
\noindent
\textbf{Google Colab utilizado:} 

\url{https://colab.research.google.com/drive/108OUqSCE_rJrN7V5GDr3DEPX329aRdW0?usp=sharing}
