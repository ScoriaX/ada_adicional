\chapter{Conclusiones}
\label{ch:con}
\section{Conclusiones}

Este proyecto tuvo como propósito principal recapitular la evolución de los algoritmos de multiplicación de enteros, haciendo énfasis en el algoritmo propuesto por Harvey y van der Hoeven, primero en alcanzar una complejidad teórica óptima de \( O(n \log n) \).

A lo largo del trabajo, se revisaron los principales métodos que quedaron marcados en la historia del problema, comenzando con la multiplicación tradicional, pasando por Karatsuba, Toom-Cook y Schöhage–Strassen, hasta llegar al avance moderno representado por Harvey–van der Hoeven; planteando la idea principal y en esencia el concepto de cada uno acompañado de su complejidad computacional, con el fin de contextualizar cada uno dentro del hito que marcaron. Reafirmando que el aun nuevo algoritmo de Harvey y van der Hoeven presentado no hace mucho en el año 2020 no solo resuelve un problema abierto por décadas, sino que sienta las bases para futuras optimizaciones.
